\documentclass{book}
\usepackage[letterpaper,top=2.5cm,bottom=2.5cm,left=2.8cm,right=2cm]{geometry}
\usepackage[toc,page]{appendix}
%\usepackage[letterpaper]{geometry}
\usepackage[table]{xcolor}
%\usepackage[draft,allpages,fontfamily=phv,color=gray!20,mark={HSA Foundation Confidential and Proprietary},fontsize=32pt,angle=45,scale=0.8,xcoord=0,ycoord=0]{draftmark}
\usepackage[fontsize=11]{scrextend}
\usepackage{makeidx}
\usepackage{natbib}
\usepackage{graphicx}
\usepackage{sidecap}
\usepackage{multicol}
\usepackage{float}
\usepackage{listings}
\usepackage{color}
\usepackage{ifthen}
\usepackage{textcomp}
\usepackage{alltt}
\usepackage{ifpdf}
\usepackage{paralist}
\ifpdf
\usepackage[pdftex,
            pagebackref=true,
            colorlinks=true,
            linkcolor=blue,
            unicode
           ]{hyperref}
\else
\usepackage[ps2pdf,
            pagebackref=true,
            colorlinks=true,
            linkcolor=blue,
            unicode
           ]{hyperref}
\usepackage{pspicture}
\fi
%\usepackage[utf8]{inputenc}
%\usepackage{mathptmx}
%\usepackage[scaled=.90]{helvet}
\usepackage{times}
\usepackage{sectsty}
\usepackage{amssymb}
\usepackage[titles]{tocloft}
\lstset{language=C++,inputencoding=utf8,breaklines=true,breakatwhitespace=true,tabsize=4,
                basicstyle=\ttfamily,
                xleftmargin=\parindent,
                keywordstyle=,
                identifierstyle=\itshape,
                emph={funcfilelist},emphstyle={\textbf},
                morekeywords={uint64_t, uint32_t, uint8_t, structfilelist},
                stringstyle=\color{red}\ttfamily,
                commentstyle=\color{brown}\ttfamily,
                morecomment=[l][\color{magenta}]{\#}
}
\usepackage{fancyhdr}
\usepackage{quotchap}
\usepackage{titlesec}
\usepackage{trackchanges}
\usepackage{enumitem}
\usepackage{framed}
\usepackage[ampersand]{easylist}
\ListProperties(Hide=100, Hang=true, Progressive=3ex, 
Style2*=$\bullet$ ,Style1*=$\triangleright$ )


\setlength{\parskip}{4mm plus2mm minus2mm}

\newcommand{\diffblock}[1]{#1}

%\DeclareFontShape{OT1}{cmtt}{bx}{n}
%     {
%      <5><6><7><8><9><10><10.95><12><14.4><17.28><20.74><24.88>cmttb10
%      }{}

\newcommand{\ttbf}[1]{\diffblock{\texttt{\textbf{#1}}}}
\newcommand{\emphld}[1]{\begin{DIFnomarkup} \emph{#1}
\end{DIFnomarkup}}
\newcommand{\dbtt}[1]{\diffblock{\texttt{#1}}}

\addeditor{AMD}
\makeindex
\setcounter{tocdepth}{3}
\renewcommand{\footrulewidth}{0.4pt}
\renewcommand{\familydefault}{\sfdefault}

\RequirePackage[normalem]{ulem} %DIF PREAMBLE
\RequirePackage{color}\definecolor{RED}{rgb}{1,0,0}\definecolor{BLUE}{rgb}{0,0,1}
%DIF PREAMBLE
\providecommand{\DIFadd}[1]{{\protect\color{blue}{#1}}} 
%DIF PREAMBLE

\providecommand{\DIFdel}[1]{{\protect\color{red}{\sout{#1}}}}
%\providecommand{\DIFdel}[1]{{\protect\color{red}{#1}}}
%\providecommand{\DIFdel}[1]{{\protect\color{red}\protect\scriptsize{#1}}}  
\newenvironment{DIFnomarkup}{}{}

%DIF PREAMBLE
%DIF SAFE PREAMBLE %DIF PREAMBLE
\providecommand{\DIFaddbegin}{} %DIF PREAMBLE
\providecommand{\DIFaddend}{} %DIF PREAMBLE
\providecommand{\DIFdelbegin}{} %DIF PREAMBLE
\providecommand{\DIFdelend}{} %DIF PREAMBLE
%DIF FLOATSAFE PREAMBLE %DIF PREAMBLE
\providecommand{\DIFaddFL}[1]{\DIFadd{#1}} %DIF PREAMBLE
\providecommand{\DIFdelFL}[1]{\DIFdel{#1}} %DIF PREAMBLE
\providecommand{\DIFaddbeginFL}{} %DIF PREAMBLE
\providecommand{\DIFaddendFL}{} %DIF PREAMBLE
\providecommand{\DIFdelbeginFL}{} %DIF PREAMBLE
\providecommand{\DIFdelendFL}{} %DIF PREAMBLE

\pagestyle{fancy}
\renewcommand{\chaptermark}[1]{ \markboth{#1}{} }
\renewcommand{\sectionmark}[1]{ \markright{#1}{} }
\renewcommand{\headrulewidth}{0.5pt}

\hfuzz=15pt
\setlength{\emergencystretch}{15pt}

\fancyhead{} % clear all header fields
\fancyhead[RO,LE]{\bfseries  HSA Core API Programmers Reference Manual}
\fancyfoot{} % clear all footer fields
\fancyfoot[LE,RO]{\thepage}
\fancyfoot[CO,RE]{HSA Core API Programmers Reference Manual}
\fancyhead[LE]{\fancyplain{}{\bfseries\thepage}}
\fancyhead[CE]{\fancyplain{}{}}
\fancyhead[RE]{\fancyplain{}{\bfseries\leftmark}}
\fancyhead[LO]{\fancyplain{}{\bfseries\rightmark}}
\fancyhead[CO]{\fancyplain{}{}}
\fancyhead[RO]{\fancyplain{}{\bfseries\thepage}}
\fancyfoot[CE]{\fancyplain{}{}}
\fancyfoot[LO,RE]{HSA Core API Programmers Reference Manual}
\fancyfoot[CO]{\fancyplain{}{}}

\hbadness=750
\tolerance=750
\begin{document}
\raggedright
%\large
\hypersetup{pageanchor=false,citecolor=blue}
\begin{titlepage}
\includegraphics[width=.5\textwidth]{foundation.png}
\vspace*{7cm}
\begin{center}
{\Large H\-S\-A Core API Programmers Reference Manual\\[1ex]\large
\change[AMD]{0.\-173}{0.\-174} }\\
\vspace*{1cm}
\vspace*{0.5cm}
{\small 17 February 2013}\\
\vspace*{0.5cm}
{\small Draft}\\
\end{center}
\end{titlepage}
\thispagestyle{empty}
{\textcopyright 2013 HSA Foundation. All rights reserved.\\}
The contents of this document are provided in connection with the
HSA Foundation specifications. The HSA Foundation makes no
representations or warranties with respect to the accuracy or
completeness of the contents of this publication and reserves the
right to make changes to specifications and product descriptions at
any time without notice. The information contained herein may be of
a preliminary or advance nature and is subject to change without
notice. No license, whether express, implied, arising by estoppel,
or otherwise, to any intellectual property rights are granted by
this publication. The HSA Foundation assumes no liability
whatsoever, and disclaims any express or implied warranty, relating
to its products including, but not limited to, the implied warranty
of merchantability, fitness for a particular purpose, or
infringement of any intellectual property right.
\clearpage
\pagenumbering{roman}
\tableofcontents
\addtocontents{toc}{~\hfill\textbf{Page}\par}
\clearpage

\pagenumbering{arabic}
\setcounter{page}{1}

%\chapter{changes} \label{changes} \hypertarget{changes}{}
%\section{Changes}
\subsection{Rev 0.15}
\begin{itemize}
\item \add[AMD] {deleted section on core graphics}
\item \add[AMD] {deleted vendor enumeration proposal}
\item \add[AMD] {replaced the existing multi-threaded queue example
with a simpler multi-threaded enqueue example on request}
\end{itemize}

\subsection{Rev 0.16}
\begin{itemize}
\item remove doxygen output usage and adding API directly to individual sections -- made it up to section 3.3
\item move architected dispatch and example to the end
\item move device to device enqueue to the architected dispatch chapter
\item remove images
\item added section on topology discovery
\item added section on AQL
\end{itemize}

\subsection{Rev 0.17}
\begin{itemize}
\item finished adding direct API instead of doxygen-included API all
the way into the document
\item updated queue structure and example to reflect what is
described in the systems architecture requirements
\item 
\end{itemize}


\hypersetup{pageanchor=true,citecolor=blue}
\chapter{Introduction} \label{index}\hypertarget{index}{}
\input{index}

\chapter{HSA Core A\-P\-I Specification and Description} \label{coreapi}
\hypertarget{coreapi}{}
This chapter describes HSA Core runtime API by their functional
area. Note that except for any setter/getter API, the remainder of
the core runtime API may be considered thread-safe.  Both the signal
update and the queue index update API are setter/getter API and
define scope an synchronization that applies to the updates and
operate on structure elements.

Several operating systems allow functions to be executed when a DLL
or a shared library is loaded (e.g. DLL main in Windows and GCC
\emph{constructor/destructor} attributes that allow functions to be
executed prior to main in several operating systems). 
Whether or not the HSA runtime functions are allowed to be invoked
in such fashion may be implementation specific and are outside the
scope of this specification.

Similarly, any header files distributed by the HSA foundation for
this spec may contain calling convention specific prefixes such as
\_\_cdecl or \_\_stdcall. Such calling conventions are again invocation,
usage and implementation specific. Hence, the calling convention
specific prefix definition is outside the scope of API definition. 

%\input{coreapierror070.tex}
%\input{coreapiopen070.tex}
%\hypertarget{component}{}\section{HSA Topology and Component
}\label{topology}
HSA platform topology information is provided by the runtime via.
data structures so user can gather details about how a HSA
system/platform decided to expose its architectural details such as
components, memory, caches and connectivity (platform topology
requirement is described in the SAR document).  This information
could be utilized by the user in different ways including decisions
on where to execute a particular user task.  Core runtime
specification defines the topology table data structure and other
data structures to represent topology hierarchy.  After the core
runtime is initialized with \ttbf{hsa\_open}, the user may create a
local copy of the topology information using the API
\ttbf{hsa\_topology\_table\_create}. The user can parse this table
representing the HSA system to gather details such as the number of
different HSA Components on the system with local access to a
particular set of memory resources.

The \ttbf{hsa\_topology\_table\_create} API is defined as follows:

\input{APItopology_create}

The API returns \dbtt{HSA\_STATUS\_SUCCESS} if the table has been
successfully created and returned via {\itshape header}. Otherwise,
it returns one of the following errors:

\begin{easylist}
& \dbtt{HSA\_STATUS\_ERROR\_INVALID\_ARGUMENT} if {\itshape header}
is NULL.

& \dbtt{HSA\_STATUS\_ERROR\_OUT\_OF\_RESOURCES} if there is a failure
in allocation of an internal structure required by the core runtime
or in the create of table header or the actual table.
\end{easylist}

\begin{figure}
  \centering
  \includegraphics[width=0.5\textwidth]{topologytable}
  \centering
  \caption{Structure of the Topology Table}
  \label{fig:topology_table}
\end{figure}

The table structure is shown in Figure~\ref{fig:topology_table}.
The first entity in the table is a table header. This is the output
of the \ttbf{hsa\_topology\_table\_create} API.
The table header is defined by the following structure:
\input{STRtopology_header}

The table header structure includes the platform structure (
\dbtt{hsa\_platform\_t}).  The platform information in the platform
structure includes size/offset-array pairs for HSA agents
(\dbtt{hsa\_agent\_t}), memory(\dbtt{hsa\_memory\_descriptor\_t} and
cache (\dbtt{hsa\_cache\_descriptor\_t}).  
HSA platform can have a hierarchical structure with multiple
components/agents and physical memories.  The
\dbtt{hsa\_platform\_t} structure also includes properties such as
the clock frequency that are common across the platform and also
links to various elements in the topology table (see Figure
\ref{fig:topology_table} ).

The platform structure is defined as follows:

\input{STRhsa_platform}

When no information is available about a particular element, the
corresponding {\itshape number\_ \textless element \textgreater s}
field is set to zero by the runtime in the platform structure.
Platform structure maps to the agents, cache and physical memory,
etc.\, in the topology table for all the nodes in the platform. 

The core runtime defines the following structure to represent cache:
\input{STRhsa_cache_descriptor}
The structure holds associativity, cache size, cache line size for
all levels of cache and the inclusivity property for all but the
last level. Each cache in the HSA system has a unique cache ID
identifying it. 

The memory descriptor structure represents a physical memory block
or region and includes elements to provide provide bandwidth, interleave
characteristics and latency for accessing memory. Implementations
may choose not to provide memory bandwidth or latency information.
The memory descriptor structure is defined as follows:

\input{STRhsa_memory_descriptor} 

The structure: 
\input{STRhsa_segment} 
can represent any combination of of the 7 HSA segments, a single
bit for each segment. 

The HSA Agent data structure represents an HSA component when the
{\itshape agent\_type} field in the agent structure is set to a 1
(i.e. bit 0 is set to 1).
The structure contains elements that describe its properties. Each
component has access to coherent global memory (the HSA global
segment, and as per the requirement defined in SAR, has access to
other segments as well). The {\itshape agent\_type} is utilized as a
bit-field. Setting bit 2 indicates that the agent is a host, bit 3
indicates that agent can participate in agent dispatches. All the
three bits or a combination of them can be set by the HSA runtime.

The structure of the HSA agent/component is defined as follows:
\input{STRhsa_component}

With in the agent, the agent type is an enumeration that is defined
as follows:
\input{ENUagent_type}

The user must destroy the topology table before closing the runtime.
The \dbtt{hsa\_topology\_table\_destroy} API is defined by the
runtime for the user to destroy the topology table. Once a table is
created, some parts of it may become invalid if any HW is
hot-plugged/unplugged or encounters an error. If such a change
occurs, the HSA runtime generates an asynchronous error (see
Section~\ref{asyncerror}) with the \dbtt{hsa\_status\_t} enumeration
of \dbtt{HSA\_ERROR\_TOPOLOGY\_CHANGE}. This is an indication to the
user that any current usage of topology table must be stopped and a
new topology table obtained by using the
\ttbf{hsa\_topology\_table\_create} API call. The runtime guarantees
that any call made to \ttbf{hsa\_topology\_table\_create} API after
the asynchronous error is observed will return the latest version of
the topology table at the time of the API invocation. However, if
the same HW was hot-swapped out and in with the same interval, or if
the error encountered in a component was recovered, the topology
table may not look different from the users perception. 


%\hypertarget{signals}{}\section{Memory based Signals and
Synchronization in H\-S\-A}\label{signals}

In a HSA system, memory is coherent and can serve as a means for
message passing, asynchronous communication or synchronization
between various elements.  A signal is an alternative, possibly more
power-efficient, communication mechanism between two entities in a
H\-S\-A system. A signal carries a value, which can be updated or
conditionally waited upon via an API call or an HSAIL
instruction~\cite{prm}. A signal structure is opaque and is
always typedef'ed to \dbtt{uint64\_t}. 
Implementations can use the most power-efficient send-propagation
and wait techniques available to them on the HSA system.  
%Implementations may chose to
%make it a pointer to a private, internal, signal structure.

The HSA SAR Specification \cite{sar} identifies HSA Agent as a
participant in a HSA memory based signaling and synchronization.
This feature, as stated in the HSA SAR Specification, requires a
runtime API for allocation of signals that may be used for
synchronization and states that the signal is opaque and may contain
implementation specific information.  

An API, \ttbf{hsa\_signal\_create}, to support
creation of signals, is defined as follows:

\input{APIsignal_create}

The signal create API returns \texttt{HSA\_STATUS\_SUCCESS} if the
signal object has been successfully created. Otherwise, it returns
one of the following:

\begin{easylist}
& \texttt{HSA\_STATUS\_ERROR\_OUT\_OF\_RESOURCES} if there is a failure
in allocation of an internal structure required by the core runtime
library in the context of the message queue creation. This error may
also occur when the core runtime library needs to spawn threads or
create internal OS-specific events. 
& \texttt{HSA\_STATUS\_ERROR\_INVALID\_ARGUMENT} if {\itshape
\diffblock{signal\_handle}} is NULL or an invalid pointer of it an invalid/NULL
context is passed in as an argument.
\end{easylist}

Once a signal is created for a particular context, it may be bound
to other contexts. This is useful when signal is used across
different components of a users application. An API to bind the
signal to a particular runtime context is defined as follows:

\input{APIsignal_bind}

This API returns \texttt{HSA\_STATUS\_SUCCESS} if the bind was
successful. Otherwise, it returns one of the following errors:

\begin{easylist}
& \texttt{HSA\_STATUS\_ERROR\_INVALID\_ARGUMENT} is {\itshape
signal\_handle} is NULL or invalid or if the {\itshape context} is
NULL or invalid. 
\end{easylist}

The corresponding signal destruction API is defined as follows:

\input{APIsignal_destroy}
The signal destroy API returns \texttt{HSA\_STATUS\_SUCCESS} if the
signal object has been successfully destroyed. Otherwise, it returns
one of the following:

\begin{easylist}
& \texttt{HSA\_STATUS\_ERROR\_INVALID\_ARGUMENT} if {\itshape
signal\_handle} is invalid.
\end{easylist}

A signal can also be unbound from a particular context if the user
no longer wants to receive notifications about this signal in the
callback registered for that context. The API to unbind is defined
as follows:

\input{APIsignal_unbind}

The API returns \texttt{HSA\_STATUS\_SUCCESS} if the signal is
successfully unbound from the context. Otherwise, it can return one
of the following errors:

\begin{easylist}
& \texttt{HSA\_STATUS\_ERROR\_SIGNAL\_NOT\_BOUND} if the signal was
not already bound to that context.

& \texttt{HSA\_STATUS\_ERROR\_INVALID\_ARGUMENT} is {\itshape
signal\_handle} is NULL or invalid or if the {\itshape context} is
NULL or invalid. 
\end{easylist}

As per the HSA SAR specification the signals may only be created and
operated on by either instructions in HSAIL or the HSA runtime API.
Sending a signal entails updating a particular value at the signal.
Waiting on a signal returns the current value at the opaque signal
object -- the wait has a runtime defined timeout which indicates the
maximum amount of time that an implementation can spend waiting for
a particular value before returning. 

The API to query the timeout is defined as:

\input{APIsignal_timeout}

This getter API does not return a status.  This API returns the
timeout, which indicates the maximum amount of time an
implementation can spend in a wait operation on the signal. The
return value is in the units of 
the system-wide clock who's frequency is available via the
\dbtt{hsa\_platform\_t} structure (see Section~\ref{topology}). As
per SAR, the HSA system has a system-wide timestamp that operates at
a fixed frequency. The frequency can be
queried via the \texttt{hsa\_platform\_t} structure defined in
Section~\ref{topology}. The timeout is incremented at the same
frequency.  The user can use this information to translate the
timeout to a different frequency domain. 

The send signal API sets the signal handle with caller specified
value. Any subsequent wait on the signal handle would be given 
a copy of this new signal value after the wait condition 
is met (and before the timeout expires).  The signal infrastructure
allows for multiple waiters on a single signal. A multi-threaded
user application can have multiple threads sending and waiting on
signals. 

In addition to the update of signals using
Send, the API for send signal must support other atomic operations as
well. HSA defines \emph {AND, OR, XOR, Exchange, Add, Subtract,
Increment, Decrement, Maximum, Minimum} and \emph{CAS}. Apart from
the no synchronization case, which is referred to as \emph{none}
synchronization, there are three types of synchronization defined in
the systems architecture requirements: 

\begin{description}
        \item[Acquire synchronization] \hfill \\ 
                No memory operation listed after the acquire can be
                executed before the acquire-synchronized operation. Acquire
                synchronization can be applied to various operations
                including a load operation.
        \item[Release synchronization] \hfill \\ 
                No memory operation listed before the release can be
                executed after the release-synchronized operation. Release
                synchronization can be applied to various operations
                including a store operation.
        \item[Acquire-Release synchronization] \hfill \\
                This acts like a fence. No memory operation listed
                before the Acquire-Release synchronized operation
                can be move after it nor can any memory operation
                listed after the Acquire-Release synchronized
                operation can be executed before it.
        \item[Relaxed synchronization] \hfill \\
                No synchronization is applied to the send or wait
                operation.
\end{description}
                
Each operation on a signal value has the type of synchronization
explicitly included in its name. For example, Send-Release is a Send
on a signal value with Release synchronization.

Hence, the following represent the complete set of actions (with
associated synchronization) that can be performed on a signal value:
Send with release, 
Send with relaxed,
AND with release,
AND with relaxed,
OR with release,
OR with relaxed,
XOR with release,
XOR with relaxed,
Exchange with acquire-release,
Exchange with relaxed,
Add with release,
Add with relaxed,
Subtract with release,
Subtract with relaxed,
Increment with release,
Increment with relaxed,
Decrement with release,
Decrement with relaxed,
Maximum with acquire-release,
Maximul with relaxed,
Minimum with acquire-release,
Minimum with relaxed,
CAS release.

For efficiency, a unique signal API has been created for each of
these actions. In the description of the API, for convenience, 
\emph{value@signal\_handle} is used to represent the value at a
signal. 

\input{APIsignal_all}

All of the \ttbf{signal\_send} API return
\texttt{HSA\_STATUS\_SUCCESS} if the send is successful, any atomic
operation that needed to be performed has been done successfully and
any result value that needs to be returned has been copied into the
user-given location. One of the following error values may be
returned in case the send is not successful:

\begin{easylist}
& \texttt{HSA\_STATUS\_ERROR\_INVALID\_ARGUMENT} if (a) the user is
expecting an output but the pointer to the output signal value is
invalid, (b) the {\itshape signal\_value} doesn't represent a valid
signal.
\end{easylist}

The user may wait on a signal, with a condition specifying the terms
of wait. The wait can be done either in the HSA Component via.\ an
HSAIL wait instruction or via.\ a runtime API defined here. 
Waiting on a signal returns the current value at the signal. The
wait may return before the condition is satisfied or even before a
valid value is obtained from the signal. It is the users burden to
check the return status of the wait API before consuming the
returned value. 

Wait \emph{reads} the value, hence Acquire and Acquire-Release
synchronizations may be applied to the read. The synchronization
should only assumed to have been applied if the status returned by
the wait API indicates a success (i.e. return type is
\texttt{HSA\_STATUS\_SUCCESS}). The two wait APIs to support both the
synchronizations are defined as follows:

\input{APIsignal_wait}

The user must always check the return value of the wait before
considering the {\itshape wait\_value} as the wait may have returned
due to a timeout. The wait API can return the following status:
\begin{easylist}
& If an error is signaled on the signal the user is waiting on, the
wait API returns \texttt{HSA\_STATUS\_ERROR} to indicate that an
error has occurred. The API still returns the current value at the
signal. The user may also inspect the value returned.
when an error occurred (see Section~\ref{signal_error}).
& \texttt{HSA\_STATUS\_ERROR\_INVALID\_ARGUMENT} if (a) the user is
expecting an output but the pointer to the output signal value is
invalid, (b) the {\itshape signal\_value} doesn't represent a valid
signal.
& \texttt{HSA\_STATUS\_INFO\_SIGNAL\_TIMEOUT} the signal wait has
timedout.
\end{easylist}

The \texttt{hsa\_wait\_condition\_t} is defined as follows:

\input{ENUwait_condition}

The runtime also defines an API to query the current signal value.
If the signal is being updated by the component or other threads,
there is no guarantee that the value returned by the query API is
the value of the signal even at the instance it has been returned.
Queried value may be used to check progress of a kernel, if the
kernel were updating the signal at various stages of its execution.
Query is a non-blocking API and does not take
\texttt{hsa\_wait\_condition\_t} as input. It merely obtains the
current value at the signal.

The \texttt{hsa\_signal\_query\_acquire} API is defined as follows:

\input{APIsignal_query} 

The \texttt{hsa\_signal\_query\_acquire} API returns
\texttt{HSA\_STATUS\_SUCCESS} when the value at the signal has been
successfully returned. Otherwise, it returns one of the following
errors:

\begin{easylist}
& \texttt{HSA\_STATUS\_ERROR\_INVALID\_ARGUMENT} if {\itshape
signal\_handle} is invalid.
\end{easylist}

Signals may be utilized in many ways. For example, a running kernel,
after it finishes producing a part of its computation, may set the
signal in the dependency packet of another kernel dispatch so that
the queue processor can resolve the dependency and launch the kernel.

Signals cannot be used for Inter-Process Communication (IPC).

\hypertarget{signal_error}{} \subsection{ Indicating Errors with
Signals} \label{signal_error}
To put the signal in error state, the two most significant bits in
the signal value are set and all other bits cleared. It is the users
burden to check to see if an error has occurred by looking at the
return code of the
\texttt{hsa\_signal\_wait<acquire\_release/Acquire>} API. Any
negative value at the signal triggers the
\texttt{HSA\_STATUS\_ERROR} return code from the wait API. A signal
that is already in error may further be decremented to a larger
negative value. 

\hypertarget{signal_example}{} \subsection{Usage Example}
\label{signal_example}

%\input{coreapiqueue070.tex}
%\input{coreapiaql070.tex}
\input{coreapierror.tex}
\input{coreapiopen.tex}
\input{coreapitopo.tex}
\input{coreapisignal.tex}
\input{coreapiqueue.tex}
\input{coreapiaql.tex}
\input{coreapimemory.tex}

\hypertarget{coreapi_coredebug}{}\section{Execution Control At the Core Level}\label{coreapi_coredebug}

As per the systems architecture specification, the H\-S\-A system must
support debugging of a H\-S\-A\-I\-L kernel. The H\-S\-A
Programmers Reference Manual (P\-R\-M) describes that the
\char`\"{}block\char`\"{} section could hold debug data and such a
section can be placed within a function. This allows the
high-\/level compiler that generates H\-S\-A\-I\-L to embed debug
specific information. This information makes its way into the
\char`\"{}.\-debug\char`\"{} section in the brig. This information
can be used for associating a H\-S\-A\-I\-L level instruction to the
higher level functionality. In addition to this, the P\-R\-M also
discusses the \ttbf{debugtrap\_u32} that halts the current wavefront
and transfers control to the agent.  The single operand to
\ttbf{debugtrap\_u32}, \char`\"{}src\char`\"{} is passed to the
agent and can be used to identify the trap.

To support this infrastructure in the runtime, the Core A\-P\-I
defines a structure that can be used to exchange information between
the kernel executing on the H\-S\-A component and the agent. 

The core runtime defines a structure, mailbox, whose purpose is to
exchange information as a part of execution control. Mailbox is a
synchronous communication mechanism between the H\-S\-A component
and any agents. The H\-S\-A component indicates a \ttbf{
debugtrap\_u32} or syscall activity by sending a signal indicating
it has written to some location in the mailbox.

The HSA PRM defines:

\begin{description}
\item \ttbf{queueactivegroupcount\_global\_u32}  {\itshape dest, address}
Returns the maximum number of work-groups that can be executed in
parallel for dispatches executed on the User Mode Queue with
address.

\item \ttbf{activegroupid} index that ranges from 0 through
\ttbf{queueactivegroupcount\_global\_u32}-1.
\end{description}

The mailbox is an array of structures of size
\ttbf{queueactivegroupcount\_global\_u32}. Since
\ttbf{activegroupid} is always unique within a queue for any
concurrent execution of kernels in that queue, indexing into the
mailbox by different work items happens without conflicts. When a
workgroup encounters a syscall or a \ttbf{debugtrap\_u32}, the
component indexes into its mailbox by accessing it via
\ttbf{activegroupid} from within the \ttbf{queueptr}. Once the
corresponding mailbox is accessed, pertinent information (see
structure below) for each work group is populated.  Subsequently the
component sets the full flag, sends a signal to agent by accessing the
{\itshape mailbox\_signal} inside the queue structure (see
Section~\ref{architected_queue}), and waits for the full flag to be
emptied. The mailbox structure is defined as follows.
\input{ENUinterrupt_condition}
\input{STRexecution_info}

The Agent waits on the signal, processes the mailbox, and clears
the full flag.

If this kernel had a debugtrap\-\_\-u32, a simple check for
debugtrap can be written the following way\-:

\begin{framed}
\lstinputlisting{mailbox_simple.c}
\end{framed}

\hypertarget{coreapi_agent}{}\section{Agent Dispatch Support at the
Core Level}\label{coreapi_agent} The core runtime supports agent
dispatches from an HSA component/Agent. The runtime defines a
default service queue for every user mode queue created by the user.
This default service queue is available to the HSAIL program HSAIL
programs and the user applications may submit agent dispatch packets
to the service queue or any user mode queue.  The service queue
shares the same structure as the regular HSA queue.  The default
service queues are monitored by the runtime.

\input{APIagent_dispatch}


\hypertarget{extensions}{}\section{Extensions to the 
Core Runtime API}\label{extensions}

When an implementor of the core runtime specification is not
supporting any of the extension API, they will return
\texttt{HSA\_STATUS\_ERROR\_EXTENSION\_NOT\_SUPPORTED} as a return
status for that API. 
 
Individual vendors may define vendor extensions to HSA core runtime,
or multiple vendors may collaborate to define an extension. The
difference is in the naming scheme used for the symbols (defines,
structures, functions, etc.\ ) associated with the function:

\begin{itemize}
\item Symbols for single-vendor extensions that are defined in the
global namespace must use the following naming convention:
  \begin{itemize}
    \item \emph{hsa\_svext\_\textless COMPANY\_NAME \textgreater\_}.
    For example, a company ``ACME'' defining a single-vendor extension
    would use the prefix \emph{hsa\_ext\_acme\_}. Company names must
    be registered with the HSA Foundation, must be unique, and may be
    abbreviated to improve the readability of the symbols. 
  \end{itemize}
\item Symbols for multi-vendor extensions that are defined in the
global namespace must use the following naming convention:
  \begin{itemize}
    \item \emph{hsa\_ext\_} For example, if another company
    embraces extension in the example above from Company ``ACME'', the
    resulting symbols would use the prefix \emph{hsa\_mvext\_}.
  \end{itemize}
\end{itemize}
 
Any constant definitions in the extension (\#define/enumerations) use
the same naming convention, except using all capital letters. So,
using the single-vendor extension example from above, the associated
defines and enumerations would have the prefix
\texttt{HSA\_EXT\_ACME\_}.
 
The symbols for all vendor extensions (both single-vendor and
multi-vendor) are captured in the file {\bf hsa/vendor\_extensions.h}.
This file is maintained by the HSA Foundation.  This file includes
the enumeration \texttt{hsa\_vendor\_extension\_t} which defines a
unique code for each vendor extension and multi-vendor extension.
Vendors can reserve enumeration encodings through the HSA
Foundation. Multi-vendor enumerations begin at the value of
1000000. For example, using the examples above, the
\texttt{hsa\_vendor\_extension\_t} enumeration might be:

\input{ENUvendor_ext}
 
HSA defines the following query function for vendor extensions:

\input{APIquery_vendorextension} 
 
This API returns \texttt{HSA\_STATUS\_SUCCESS} if the extension is
supported.  Additionally, {\itshape extension\_structure} is written
with extension-specific information such as version information,
function pointers, and data values.  {\bf hsa/vendor\_extension.h} defines
a unique structure for each extension.  If the vendor extension is
not supported, \texttt{HSA\_STATUS\_ERROR\_EXTENSION\_UNSUPPORTED}
is returned, and \texttt{extension\_structure} is not modified.

\subsection{Example Definition And Usage of an Extension} 
An example that shows a hypothetical single-vendor extension ``Foo''
registered by company ``ACME''.  The example includes four defines
and two API functions.  Note the use of the structure
\texttt{hsa\_svext\_acme\_foo\_t} and how this interacts with the
\ttbf{hsa\_query\_vendor\_extension} API call.

\lstinputlisting{extension.c}
 
%%%%%%%%%%%%%%%%%%%%%%%%%%%%%%%%%%%%%%%%%%%%%%%%%%%%%%%%%%%%%%%%%%%%%%%%%%
%%%%%%%%%%%%%%%%%%%%%%%%%%%%%%%%%%%%%%%%%%%%%%%%%%%%%%%%%%%%%%%%%%%%%%%%%%
%An enumeration, \texttt{HsaSignalSyncType} allows the user to specify the
%synchronization type for a particular send or wait operation on a
%signal. It is defined as follows:
%
%\begin{framed}
%  \begin{lstlisting}
%    typedef enum {
%            kHsaNone=0,
%            kHsaAcquire=1,
%            kHsaRelease=2,
%            kHsaAcquireRelease=3
%    }HsaSignalSyncType ;
%  \end{lstlisting}
%
%\diffblock{
%  \begin{description} [font=\tt]
%    \item[kHsaNone] \hfill \\
%            indicates that none of the below synchronization methods
%            are desired
%    \item[kHsaAcquire] \hfill \\
%            acquire, applies to wait and atomic send
%    \item[kHsaRelease] \hfill \\
%            release, applies to send and atomic send
%    \item[kHsaAcquireRelease] \hfill \\
%            applies to send, wait and atomic send
%  \end{description}
%}
%
%\end{framed}

%An enumeration, \texttt{HsaSignalAction} allows the user to
%specify what to do with the \emph{value} in the
%\texttt{HsaSignalSend} API. The enumeration is defined as follows:
%
%\begin{framed}
%  \begin{lstlisting}
%    typedef enum {
%            kHsaSignalSet=0,
%            kHsaSignalAtomicAnd=1,
%            kHsaSignalAtomicOr=2,
%            kHsaSignalAtomicXor=3,
%            kHsaSignalAtomicExch=4,
%            kHsaSignalAtomicAdd=5,
%            kHsaSignalAtomicSub=6,
%            kHsaSignalAtomicInc=7,
%            kHsaSignalAtomicDec=8,
%            kHsaSignalAtomicMax=9,
%            kHsaSignalAtomicMin=10,
%            kHsaSignalAtomicCas=11
%    }HsaSignalAction ;
%  \end{lstlisting}
%\diffblock{
%  \begin{description} [font=\tt]
%    \item[kHsaSet] \hfill \\
%            basic set signal with release, no atomics 
%    \item[kHsaSignalAtomicAnd] \hfill \\
%            atomic AND, signal.value \&= value
%    \item[kHsaSignalAtomicOr] \hfill \\
%            atomic OR, signal.value OR= value
%    \item[kHsaSignalAtomicXor=3,] \hfill \\
%            atomic XOR, signal.value XOR= value
%    \item[kHsaSignalAtomicExch] \hfill \\
%            atomic Exch
%    \item[kHsaSignalAtomicAdd] \hfill \\
%            atomic add, signal.value += value
%    \item[kHsaSignalAtomicSub] \hfill \\
%            atomic subtract, signal.value -= value
%    \item[kHsaSignalAtomicInc] \hfill \\
%            atomic increment, signal.value++, value is ignored
%    \item[kHsaSignalAtomicDec] \hfill \\
%            atomic increment, signal.value--, value is ignored
%    \item[kHsaSignalAtomicMax] \hfill \\
%            atomic maximum, MAX(signal.value,value) 
%    \item[kHsaSignalAtomicMin] \hfill \\
%            atomic minimum, MIN(signal.value,value) 
%    \item[kHsaSignalAtomicCas] \hfill \\
%            atomic compare and swap, if(signal.value == value2)
%            signal.value = value2
%  \end{description}
%}
%\end{framed}

% \input{signal_send_release}
% \input{signal_send_acquire_release}
% \input{signal_and_release}
% \input{signal_or_release}
% \input{signal_xor_release}
% \input{signal_exch_release}
% \input{signal_exch_acquire_release}
% \input{signal_add_release}
% \input{signal_sub_release}
% \input{signal_inc_release}
% \input{signal_dec_release}
% \input{signal_max_none}
% \input{signal_min_none}
% \input{signal_cas_release}

%To just signal using a set (no atomics), action
%\texttt{kHsaSignalSet} may be used. However, only
%\texttt{kHsaRelease} and \texttt{kHsaAcquireRelease} synchronization
%types apply to the \texttt{kHsaSignalSet} action. 

%The Table \ref{actionandsync} shows which synchronization operations
%apply to which signal actions. The implementation of the
%\texttt{HsaSignalSend} will return a failure if correct combinations
%are not used.

% \begin{table}[b!]
% \begin{center}
%         \begin{tabular}{|p{3in}| p{3in}|}
%     \hline
%     \textbf{HsaSignalAction} & \textbf{Correponding HsaSignalSyncType that can be
%     used with the action} \\
%     \hline
%     kHsaSignalSet & kHsaRelease, kHsaAcquireRelease \\ \hline
%     kHsaSignalAtomic<And,Or,Xor,Exch, Add,Sub,Inc,Dec,Max,Min,Cas &
%     kHsaRelease, kHsaNone, kHsaAcquire, kHsaAcquireRelease \\ \hline
%   \end{tabular}
% \end{center}
% \caption{Action and Synchronization Combinations}
% \label{actionandsync}
% \end{table}

% \begin{framed}
%   \begin{lstlisting}
%   typedef enum {
%           kHsaSignalConditionEqual, 
%           kHsaSignalConditionNotEqual,
%           kHsaSignalConditionLessThan,
%           kHsaSIgnalConditionGreaterThanOrEqual
%   }HsaSignalWaitCondition;
%   \end{lstlisting}
% \diffblock{
%   \begin{description}[font=\tt]
%     \item[kHsaSignalConditionEqual] \hfill \\
%             wait until timeout or value == signal.value
%     \item[kHsaSignalConditionNotEqual] \hfill \\  
%             wait until timeout or value $\ne$ signal.value
%     \item[kHsaSignalConditionLessThan] \hfill \\
%             wait until timeout or value \textless  signal.value
%     \item[kHsaSignalConditionGreaterThanOrEqual] \hfill \\
%             wait until timeout or value $\ge$ signal.value
%   \end{description}
% }
% \end{framed}

% In addition to this, the wait API must support \texttt{kHsaRelease}
% and \texttt{kHsaAcquireRelease} synchronizations. 

%\begin{framed}
%\lstinputlisting{aql_dispatch.c}
%\end{framed}



\begin{appendices}
\chapter{Compilation Unit, Finalizer and ISA Linking}
\label{finalizerchapter} \hypertarget{finalizerchapter}{}
\hypertarget{finalizer}{}\section{Finalization, Compilation Unit,
Code, Debug and Symbol Objects}\label{finalizer}

Compilation support in the HSA core runtime comprises of a
finalization step. The primary functionality of this step is to
translate the HSAIL to a component specific instruction set and
produce a compilation unit. It resolves user defined symbols that
are not already bound via callbacks provided by the user.

HSA components may support HSAIL natively or may have a native ISA
that the HSAIL needs to be translated into. However, compilation is a
necessary step and is required despite a HSA components' native support
of HSAIL. This is because of two primary reasons:

\begin{enumerate}
\item  The HSA kernel code objects (accessible via the
\texttt{hsa\_compilationunit\_code\_t} structure, discussed in
Section~\ref{finalize:codeobject} ) are an output of the
finalization process and required as an input for the AQL dispatch
packet; to obtain values from the finalizer for segment sizes etc.\
and also to act as a container for component specific execution
information.

\item In order to support kernel dispatches from the HSA
component, the kernel code object must reside in a memory layout
specification since dispatches initiated from components are able
to use memory operations to get the information necessary for a
dispatch.
\end{enumerate}

Before discussing BRIG, etc., a constant \texttt{HSA\_CODE\_VERSION}
is defined by the runtime to represent the version of the code
object formats being used in this instance of the runtime.

\begin{lstlisting}
typedef uint32_t hsa_code_version32_t;
enum hsa_code_version_t {
  HSA_CODE_VERSION = 0
};
\end{lstlisting}

\subsection{BRIG and .directive Section}
The core runtime accepts H\-S\-A\-I\-L programs coded in the
B\-R\-I\-G binary format, as defined in the HSA PRM~\cite{prm}, for its
finalization process. BRIG is a binary format defined by the HSA
PRM and includes 5 different sections, \emph{.string},
\emph{.directive}, \emph{.code}, \emph{.operand}, and \emph{.debug}.
More information on these sections is described in Section 19.1 of
the HSA PRM. The core runtime structure that represents a
BRIG is named \texttt{hsa\_brig\_t}. This structure is an in-memory
representation of the BRIG. It is defined as follows:

\input{example/STRhsa_brig}

Of the different sections in BRIG, the .directive section provides
information to the finalizer on functions, kernels, and global
declarations, etc. The symbols in the .directive section have defined
placement rules (see PRM for more information). For example,
immediately after a function or kernel directive, BRIG requires the
directives that describe the arguments to be in a certain order.
Return arguments are first, followed by input arguments, followed by
the directives that apply only to the function or kernel.

The \texttt{hsa\_brig\_t} structure has the base address of the
directive section. A directive for any symbol is represented using
an offset into the directive section. HSA core runtime defines a
type \texttt{hsa\_brig\_directive\_offset\_t} to represent the
.directive section offset. It is typedef to an unsigned 32 bit
integer and is defined by all implementations as follows:

\begin{lstlisting}
typedef uint32_t hsa_brig_directive_offset_t;
\end{lstlisting}

There are different types of directives specified in the PRM. Of
these, the control directives are a means to allow implementations
to pass information to the finalizer via HSAIL. HSA runtime
defines a structure \texttt{hsa\_control\_directives\_t} to
represent the values of control directives both at finalization
time and to record information in the kernel code object. Control
directives may also be specified within the HSAIL code. When
conflicting values are specified for a particular directive
specified in HSAIL and at finalization time, the runtime will do one
of the following: (a) perform a union (OR operation) when possible
(e.g. when the value represents a bit field).

(b) when a union is not meaningful, the runtime will require that the value provided at
finalization time via the \texttt{hsa\_control\_directives\_t}
structure match the value for this directive in the HSAIL kernel.

The \texttt{hsa\_control\_directives\_t} structure is defined as
follows:

\input{example/STRcontrol_directive}

Where, the \texttt{hsa\_control\_directive\_present64\_t} is defined
as a 64bit unsigned integer.

\begin{lstlisting}
typedef uint64_t hsa_control_directive_present64_t;
\end{lstlisting}

The enumeration \texttt{hsa\_control\_directive\_present64\_t} is a
bit set indicating which control directives have been specified. It
is accessible via a mask,
\texttt{hsa\_control\_directives\_present\_mask\_t}, which
is defined as follows:

\input{example/ENUdirective_present}

User can choose to either break on exceptions or just detect them.
The PRM defines the policy to be exception-type specific, i.e.\ different
IEEE exceptions supported by HSA (see the definition of the
enumeration \texttt{hsa\_exception\_kind\_mask\_t} below)  can be
handled with different policies (BREAK vs.\ DETECT).
The {\itshape enable\_break\_exceptions} field specifies the set of
HSAIL exceptions that must have the BREAK policy enabled. It is
possible that on some systems, enabling exceptions may result in
lower code performance.
If the kernel being finalized has any \texttt{enablebreakexceptions}
control directives in HSAIL, then the runtime performs a union (OR
operation) of values specified by this argument with the values in
HSAIL control directives. If any of the functions the kernel calls
have an enablebreakexceptions control directive, then they must be
equal to or a subset of this union.

\input{example/ENUexception_type}

\subsection{Code Objects}\label{finalize:codeobject}

There are different code objects defined by the HSA core runtime
specification in support of HSAIL:
\texttt{hsa\_compilationunit\_code\_t},
\texttt{hsa\_kernel\_code\_t} and \texttt{hsa\_function\_code\_t}.
All of them are in memory and can be relocatable and/or position
independent (which indicates that the object can be deep-copied to
other memory locations for execution). The HSA runtime provides a
query API to verify if a particular component supports position
independent code objects (see Section~\ref{topology}).

The \texttt{hsa\_compilationunit\_code\_t} is the header for the
code object produced by the Finalizer and contains information that
applies to all code entities in the compilation unit.

Since core runtime does not define a file format container, the core
runtime provides API to work with HSAIL programs encoded in the BRIG
binary format and supports generation of code objects that
include kernels to be executed in the HSA component and binding
of unresolved symbols associated with the code object
generated.

Finalizer allocates a single contiguous area of memory to hold the
generated code for all the code objects.

The structure of this contiguous area is as follows: Starting at
offset 0, the ``header'' of this contiguous area is defined by the
\texttt{hsa\_compilationunit\_code\_t} structure. The
\texttt{hsa\_compilationunit\_code\_t} structure in turn contains
an offset to an array of \texttt{hsa\_code\_entry\_t}, one entry per
code entity that the finalizer has produced code for. Each
\texttt{hsa\_code\_entry\_t} variable contains an offset to a
\texttt{hsa\_*code\_t} object that describes that code entity
(function/kernel/etc).

The kinds of code objects that can be contained in a
\texttt{hsa\_compilationunit\_code\_t} is defined by the following
structure:
\input{example/ENUhsa_codekind}

The \texttt{hsa\_code\_entry\_t} structure is defined as follows:
\input{example/STRhsa_codeentry}

The current version number of the HSA code object
format is defined as follows:

\input{example/ENUhsa_codeversion}

Every \texttt{hsa\_*\_code\_t} code objects start with a common
header that also contains what kind of code object it is. The common
header, \texttt{hsa\_code\_t}, is defined as follows:
\input{example/STRhsa_codeheader}

A bit set of flags providing information about the code in a
compilation unit. Unused flags must be 0. The
\texttt{hsa\_code\_properties32\_t} must be used as a type for this
flag. The values/mask currently supported is defined as follows:
\input{example/ENUcodeproperties}

The \texttt{hsa\_compilationunit\_code\_t} structure is defined as
follows:
\input{example/STRhsa_compilationunit}

Both the \texttt{hsa\_compilationunit\_t} and
\texttt{hsa\_*\_code\_t} objects can have implementation define data
towards the end of the structure. All elements in the contiguous
location being accessed by offsets enables such definition.  This
also allows the exact position of the various objects in the
contiguous memory area to be implementation defined as long as
memory alignment requirements are met.

Many of the \texttt{hsa\_*\_code\_t} objects include a size field
which is required to be set to the size of the structure. This
allows forward compatibility and allows for structure definitions to
change or include implementation specific information.

The \texttt{hsa\_kernel\_code\_t} is required for all component
dispatches and is referred to by the dispatch AQL packets placed in
the HSA queue. This allows an implementation to rely on all such
objects being the same size for more efficient navigation to the
implementation specific data without need to first read the object's
size field.

The \texttt{hsa\_kernel\_code\_t} is the output of
\ttbf{hsa\_finalize\_brig} API and is defined as follows.

\input{example/STRcode_object}

\subsection{Finalize BRIG API}
The \ttbf{hsa\_finalize\_brig} API accepts \texttt{hsa\_brig\_t} as
an input and produces relocatable code object in which global and
group symbols are bound to actual, component recognizable,
addresses.  A symbol in the finalization step can be a variable, a
function, image or a sampler. When finalization occurs, a
\texttt{hsa\_kernel\_code\_t}, representing the kernel that needs to
be executed on the component is generated.  However, all the symbols
referenced in the kernel being finalized may not be resolved with
the information provided at its finalization. User is allowed to
define callbacks that can resolve the symbols declared in the global
segment.

The callback, \texttt{hsa\_map\_symbol\_address\_t}, is defined as
follows:
\begin{lstlisting}
typedef hsa_status_t (*hsa_map_symbol_address_t)(
  hsa_finalize_compilationunit_caller_t caller,
  hsa_brig_directive_offset_t symbol_directive,
  uint64_t* address);
\end{lstlisting}

The finalization step, when successfully executed, can have two
distinct outputs. The first is the compilation unit code object,
\texttt{hsa\_compilationunit\_code\_t} which is discussed in
Section~\ref{finalize:codeobject}. The second output is of type
\texttt{hsa\_compilationunit\_debug\_t} which is only generated when
the code is compiled with a debug option. This debug information is
currently implementation defined.

Since the contiguous memory that {\itshape
compilationunit\_code} represents and the memory for {\itshape
compilationunit\_debug} need to be allocated, the user is expected
to provide callbacks for memory allocation.

The callback is required to have the following signature:

\begin{lstlisting}
typedef hsa_status_t (*hsa_alloc_t)(
  hsa_finalize_brig_caller_t caller,
  size_t byte_size,
  size_t byte_alignment,
  void** address);
\end{lstlisting}

Where
caller is the opaque pointer passed to the
\ttbf{hsa\_finalize\_brig} that is calling back this function.
{\itshape byte\_size} is the size in bytes of the memory to be allocated.
{\itshape byte\_alignment} is the required byte alignment of the memory
allocated. Must be a power of 2.
{\itshape address} is pointer to location that will be updated with address of
allocated memory if successful, or NULL if not successful.
{\itshape return value} is the HSA status of the allocation.

The callback for the
allocation of {\itshape compilationunit\_debug} is optional and is
required when the user passes in a flag at finalization to generate
debug information. This callback, when defined, must have the same
signature as \texttt{hsa\_alloc\_t} above.

\input{example/APIfinalize_brig}

The \ttbf{hsa\_finalize\_brig} API returns
\texttt{HSA\_STATUS\_SUCCESS} when the finalization is successful.
Otherwise, it returns one of the following errors:

\begin{easylist}

& \texttt{HSA\_STATUS\_ERROR\_INVALID\_ARGUMENT} If {\itshape brig}
is NULL or invalid, or if {\itshape kernel\_directive} is invalid.

& \texttt{HSA\_STATUS\_INFO\_UNRECOGNIZED\_OPTIONS} If the options
are not recognized, no error is returned, just an info status is
used to indicate invalid options.

& \texttt{HSA\_STATUS\_ERROR\_OUT\_OF\_RESOURCES} If the finalize API
cannot allocate memory for {\itshape compilationunit\_code/debug} or the
deserialize cannot allocate memory for {\itshape code\_object}.

& \texttt{HSA\_STATUS\_ERROR\_DIRECTIVE\_MISMATCH} If the directive
in the control directive structure and in the HSAIL kernel mismatch
or if the same directive is used with a different value in one of
the functions used by this kernel.
\end{easylist}

The \texttt{hsa\_code\_object\_t} is defined as follows:

\input{example/STRcontrol_directive}
\input{example/STRcode_object}

%\hypertarget{coreapi__finalizer_specifying_kernel}{}\subsection{Specifying
%Kernel Reference}\label{coreapi__finalizer_specifying_kernel}
The user must ensure that the B\-R\-I\-G is valid, failing which the
finalize API will return an error.  The desired kernel to finalize
is specified as a location in the B\-R\-I\-G. The location is
described as an offset into the B\-R\-I\-G .directive section.

H\-S\-A\-I\-L does not define a set of options that a finalizer
needs to specify. To ensure portability, \ttbf{hsa\_finalize\_brig}
must not return an error status if a given compilation option is not
recognized.

\subsection{Freeing The Compilation Unit Code/Debug Object}

The core runtime also provides a corresponding destruction API that
destroys the compilation unit code and debug objects.  This will
reclaim all memory used by the
\texttt{hsa\_compilationunit\_code\_t} (including the ISA it
contains) and associated \texttt{hsa\_compilationunit\_debug\_t}. It
will also unregister the ISA memory if appropriate.

\input{example/APIfinalize_destroy}

The \ttbf{hsa\_compilationunit\_code\_destroy} API destroys the
compilation unit code object where as
\ttbf{hsa\_compilationunit\_debug\_destroy} destroys the compilation
unit debug object.

Both API return \texttt{HSA\_STATUS\_SUCCESS} if the destruction was
successful. Otherwise, they can return one of the following errors:

\begin{easylist}

& \texttt{HSA\_STATUS\_ERROR\_RESOURCE\_FREE} if some of the
resources consumed during initialization by the runtime could not be
freed.

& \texttt{HSA\_STATUS\_ERROR\_INVALID\_ARGUMENT} if {\itshape
code\_object} or {\itshape debug\_object} is NULL or does not point
to a valid code object.

\end{easylist}

Note that destroying does not impact any memory segments that may
have been allocated/reserved for use in a kernel from this object.
It merely releases resources used to build the object.

\subsection{Serializing and Deserializing a Compilation Unit}

Because of the opaque nature of what the compilation unit actually
represents, and in order for the users of the core runtime to build
file containers, serialization and deserialization API are defined
by the core runtime. The definition is as follows:

\input{example/APIfinalize_serial}

\texttt{HSA\_STATUS\_SUCCESS} is returned if serialize/deserialize
API finish successfully. Success of serialize API means the
{\itshape code\_object} has been successfully serialized and then
copied into the location in memory ({\itshape serialized\_object}).
A successful deserialization recreates the code object, allocates
memory for it and returns it. One of the following errors may be
returned:
\begin{easylist}
& \texttt{HSA\_STATUS\_ERROR\_INVALID\_ARGUMENT} If {\itshape
code\_object} or is NULL or does not point to a valid code object in
the serialize API. For the deserialize API, this means the {\itshape
serialized\_object} is either null or is not valid or the size is 0.

& \texttt{HSA\_STATUS\_ERROR\_OUT\_OF\_RESOURCES} If the serialize API
cannot allocate memory for {\itshape serialized\_object} or the
deserialize cannot allocate memory for {\itshape code\_object}.
\end{easylist}

%\subsection{Query Functions in Support of the}
%\texttt{hsa\_finalize\_brig} API

\hypertarget{coreapi_group_mem}{}\section{Group Memory
Usage}\label{coreapi_group_mem}
Group memory can be allocated either statically when the finalizer
is called or dynamically by the user at launch time.

Static allocation is supported as follows\-:

\begin{itemize}

\item The \ttbf{hsa\_finalize\_brig} routine writes the amount of
group memory needed by the finalized I\-S\-A to the
\texttt{hsa\_code\_object\_t}.{\itshape workgroup\_group\_segment\_size\_byte}
field.  The group memory usage includes group memory which is
statically allocated in the H\-S\-A\-I\-L kernel, as well as private
group memory used by the finalizer. Different H\-S\-A
implementations might allocate different amounts of group memory.

\item The user copies the requested group segment usage to the
A\-Q\-L dispatch packet's \texttt{hsa\_aql\_dispatch\_packet\_t}.{\itshape
group\_segment\_size\_bytes} field.

\item The packet processor reads the group memory usage field and
reserves the required resources at dispatch time.

\item Statically allocated group memory starts at a segment offset
of 0.

\end{itemize}

Dynamically allocated group memory allows the user to specify the
group memory size when the kernel is launched. This is useful to
support dynamic group memory allocation features supported by
languages such as Open\-C\-L. Essentially, the user manually
calculates the offset for each kernel argument (including the static
allocation in the calculation) and passes these as arguments to the
H\-S\-A\-I\-L kernel. Specifically\-:

\begin{itemize}

\item As above, the {hsa\_finalize\_brig} routine returns the
requested static group allocation.

\item H\-S\-A\-I\-L will use standard 32-\/bit arguments (that is,
\ttbf{kernarg\_u32}) to specify group segment offsets. The user is
responsible for computing the offset for each group memory argument
location. The first argument must start just above the static
allocation, so it always has the offset of
\texttt{hsa\_code\_object\_t}.{\itshape
workgroup\_group\_segment\_size\_byte}

\item After setting the offset for each group memory argument, the
user must set the \-Q\-L dispatch packet's
\texttt{hsa\_aql\_dispatch\_packet\_t}.{\itshape
group\_segment\_size\_bytes} field to the total amount of group
memory used (static and dynamic allocations).

\end{itemize}

See below for an example of setting up dynamic group memory
arguments for a kernel.
\begin{framed}
\lstinputlisting{example/group_sample.c}
\end{framed}

Here is the corresponding kernel and usage model\-:

\begin{framed}
\lstinputlisting{example/group_sample_hsail.c}
\end{framed}



\chapter{Images and Samplers}
\label{images} \hypertarget{images}{}
\hypertarget{Images}{}\section{Images} \label{images}


\chapter{Component Initiated Dispatches} \label{architected}
\hypertarget{architectedchptr}{}
\input{architected_dispatch}

\chapter{Error Status Structure and Defined Values}
\label{errstatus} \hypertarget{errstatus}{}
\hypertarget{coreapi_dtde}{}\section{Error States in Core API and
Extentions} \label{coreapi_dtde}


\end{appendices}

\bibliographystyle{plain}
\bibliography{corebase}
\end{document}
