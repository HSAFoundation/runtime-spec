\begin{DIFnomarkup}
\hypertarget{init}{}\section{Open and close
API}\label{init}
\end{DIFnomarkup}

Since HSA core runtime is a user mode library, its state is a part
of the application's process space. When the runtime is opened for
the first time, a runtime instance for that application process is
created. Closing a runtime destroys this instance. An application
may open (or close) the HSA runtime multiple times with-in the same
process and potentially with in multiple threads -- only a
single instance of the runtime, per-process, will exist. 

The core runtime defines a runtime context that acts as a reference
counting mechanism and a scheme to differentiate multiple usages of
the runtime within the same application process. The runtime context
is generated when the runtime is opened for the first time or when a
user calls the acquire API that is defined in this Section. As an
example, consider an application that is using the runtime also uses
a library that creates queues and submits work to them. Both the
library and the application want to register callbacks, but to
capture notifications/errors of their specific usage. Context helps
identify the different users (with in the same process) and channel
errors and notifications to appropriate callbacks. It also acts as a
reference counting mechanism; while correctly \emphld{acquired}, the
runtime context ensures that the runtime instance will not be
shutdown until the context is \emphld{released}.

This section defines four new API, \ttbf{hsa\_open} to open the
runtime instance, \ttbf{hsa\_close} to close it,
\ttbf{hsa\_context\_acquire} to create a new context (and increments the
reference count), and,
\ttbf{hsa\_context\_release} to release the acquired context.

Invocation of \ttbf{hsa\_open} initializes the HSA runtime if it is
already not initialized. It is allowed for applications to invoke
\ttbf{hsa\_open} multiple times and do multiple \ttbf{hsa\_close}
API calls. Any subsequent initialization of the HSA runtime library,
while it is still initialized, is allowed. The runtime
implementation exposes a reference counting mechanism to the user
via the runtime context. Reference counting is a mechanism that
allows the runtime to keep a count of the number of different
libraries or threads using the runtime API.  This ensures that the
runtime stays active until a \ttbf{hsa\_close} is called by the user
when the reference count represented by the runtime context is 1.

The definition of the \ttbf{hsa\_open} API is as follows:

\input{APIhsa_initialize}

The open API returns \dbtt{HSA\_STATUS\_SUCCESS} if the
initialization was successful. Otherwise it returns one of the
following errors:

\begin{easylist}
& \dbtt{HSA\_STATUS\_ERROR\_OUT\_OF\_RESOURCES} if there is a
failure in allocation of an internal structure required by the core
runtime library. This error may also occur when the core runtime
library needs to spawn threads or create internal OS-specific
events. 

& \dbtt{HSA\_STATUS\_ERROR\_COMPONENT\_INITIALIZATION} if there
is a non-specific failure in initializing one of the components. 

& \dbtt{HSA\_STATUS\_ERROR\_CONTEXT\_NULL} if the context pointer
passed by the user is NULL. User is required to pass in a memory
backed context pointer.
\end{easylist}

If the HSA runtime is already initialized, an asynchronous
notification is generated by the runtime and
\dbtt{HSA\_STATUS\_SUCCESS} is returned. It is the users burden to
define a callback that would potentially invoke the \emphld{acquire}
API on reference counting as multiple \ttbf{hsa\_open} calls do not
automatically increment the reference count.

The runtime defines \ttbf{hsa\_close} as the corresponding API call
to finalize the use of the runtime API. This API does not take in
any input. This API updates the reference count once upon its first
invocation. Once the reference count is 0, it proceeds to relinquish
any resources allocated for the runtime and closes the runtime
instance.  It is possible in a multi-threaded scenario that one
thread is doing a close while the other is trying to acquire the
runtime context. The core runtime implementation handles such
scenarios by defining that an acquire on context that represents a
closed runtime instance will fail. The API for \ttbf{hsa\_close} is
defined as follows:

\input{APIhsa_close}

The close API returns \dbtt{HSA\_STATUS\_SUCCESS} if the close
was successful. Otherwise, it returns one of the following errors:

\begin{easylist}
& \dbtt{HSA\_STATUS\_ERROR\_NOT\_INITIALIZED} if the close was
called (a) either before the runtime was every initialized, or (b)
after it has already been successfully closed. 

& \dbtt{HSA\_STATUS\_ERROR\_RESOURCE\_FREE} if some of the
resources consumed during initialization by the runtime could not be
freed. 
\end{easylist}

Once runtime is opened and a context obtained, user can control its
reference counting and generate new contexts. For example, if an
\ttbf{hsa\_open} call resulted in an asynchronous notification that
a prior open successfully initialized the runtime, and the user
callback implementation catches this notification, the user's
callback can explicitly request an acquire on the context.

The HSA core runtime API for an acquire on a context,
\ttbf{hsa\_context\_acquire}, is defined as follows:

\input{APIacquire_context}

The open API returns \dbtt{HSA\_STATUS\_SUCCESS} if the acquire was
successful and if {\itshape output\_context} holds the new context
generated. Otherwise it returns one of the following errors:

\begin{easylist}
& \dbtt{HSA\_STATUS\_ERROR\_NOT\_INITIALIZED} if the
\ttbf{hsa\_acquire\_context} was called (a) either before the
runtime was every initialized, or (b) after it has already been
closed. 
\end{easylist}

The corresponding release API, \ttbf{hsa\_context\_release} is
defined as follows:

\input{APIrelease_context}

The \ttbf{hsa\_context\_release} API returns
\dbtt{HSA\_STATUS\_SUCCESS} if the release was successful.
Otherwise it returns one of the following errors:

\begin{easylist}
& \dbtt{HSA\_STATUS\_ERROR\_NOT\_INITIALIZED} if the
\ttbf{hsa\_release\_context} was called (a) either before the
runtime was after reference count has already reached a value of 0.
\end{easylist}

