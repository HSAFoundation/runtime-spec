\hypertarget{init}{}\section{Initialize and Finish
API}\label{init}
Since HSA core runtime is a user mode library, its state is a part
of the application's process space. An application may initialize
and finish the HSA runtime multiple times with-in the same process
context and potentially with in multiple threads -- only a single
instance of the runtime, per-process, will exist.  Initialization of
the HSA core runtime is required for an application to use the
runtime API calls. The application is also expected to finish the
runtime once it no longer needs to use the runtime API. Calling any
runtime API function without initializing it results in an undefined
behavior.

Invocation of \ttbf{hsa\_initialize} initializes the HSA runtime
if it is already not initialized. It is allowed for applications to
initialize core runtime multiple times and do a matching set of
finishes. Any subsequent initialization of the HSA runtime library,
while it is still initialized, is allowed. The runtime
implementation relies on internal reference counting. Reference
counting is a mechanism that allows the runtime to keep an internal
count of the number of calls to \ttbf{hsa\_initialize}. This ensures
that the runtime stays active until the last \ttbf{hsa\_finish}
call. 

An error message queue is an output of the \ttbf{hsa\_initialize}
API. This error message queue serves as the \emph{default} error
message queue for the runtime (this is discussed again in
Section~\ref{architected_queue}). All invocations of the
\ttbf{hsa\_initialize} return the same default error message queue
handle, i.\ e.\ there is only one default error message queue per
instance runtime.  This API does \emph{not} return a failure when no
HSA components were discovered by the runtime. The definition of the
API is as follows:

\input{APIhsa_initialize}

The initialize API returns \dbtt{HSA\_STATUS\_SUCCESS} if the
initialization was successful. Otherwise it returns one of the
following errors:

\begin{easylist}
& \dbtt{HSA\_STATUS\_INFO\_ALREADY\_INITIALIZED} when the core
runtime library has already been initialized.

& \dbtt{HSA\_STATUS\_ERROR\_OUT\_OF\_RESOURCES} if there is a
failure in allocation of an internal structure required by the core
runtime library. This error may also occur when the core runtime
library needs to spawn threads or create internal OS-specific
events. 

& \dbtt{HSA\_STATUS\_ERROR\_COMPONENT\_INITIALIZATION} if there
is a non-specific failure in initializing one of the components. 

& \dbtt{status \textgreater \, HSA\_STATUS\_OTHER\_BEGIN} Any
implementation specific error has a error value \textgreater
\dbtt{HSA\_STATUS\_OTHER\_BEGIN} (see~\ref{error} for details).
\end{easylist}

The runtime defines \ttbf{hsa\_finish} as the corresponding API call
to finalize the use of the runtime API. This API does not take in
any input. This API merely updates the reference count until the
reference count indicates that number of initializations has been
matched with the number of finishes. Once this match is determined,
the runtime proceeds to freeing resources allocated during
initialization. It is possible in a multi-threaded scenario that one
thread is doing a finish while the other is trying to initialize.
The core runtime implementation handles such scenarios. The API is
defined as follows:

\input{APIhsa_finish}

The finish API returns \dbtt{HSA\_STATUS\_SUCCESS} if the finish
was successful. Otherwise, it returns one of the following errors:

\begin{easylist}
& \dbtt{HSA\_STATUS\_INFO\_REFCOUNT\_NONZERO} if the number of times
initialize has been called is more than the number of time finish
has been called.

& \dbtt{HSA\_STATUS\_ERROR\_NOT\_INITIALIZED} if the finish was
called (a) either before the runtime was every initialized, or (b)
after it has already been finished. 

& \dbtt{HSA\_STATUS\_ERROR\_RESOURCE\_FREE} if some of the
resources consumed during initialization by the runtime could not be
freed. 

& \dbtt{status \textgreater \, HSA\_STATUS\_OTHER\_BEGIN} Any
implementation specific error has a error value \textgreater
\dbtt{HSA\_STATUS\_OTHER\_BEGIN} (see~\ref{error} for details).
\end{easylist}

