\begin{DIFnomarkup}
\hypertarget{error}{}\section{Synchronous and Asynchronous Errors and
Asynchronous Notification}
\label{error}
\end{DIFnomarkup}

Error handling in the core runtime can broadly be classified into
two categories: synchronous error handling and asynchronous
error/notification handling. 

Synchronous errors are always reported when the call returns. They
indicate if the API returned a success or an error.

Asynchronous errors can occur due to various reasons:
\begin{inparaenum}[(i)]

\item Activity in packet processor, executing kernels, their actions and memory accesses. If an error is detected during execution of a kernel, the completion signal (if present) will be signaled with an error indication value.

\item To provide \emphld{information/warning}
(not as an exception in expected behavior but by definition). This information/warning may not necessarily indicate an error. For example, a timeout may be an acceptable response for a wait API but
is not indicative of a failure. \end{inparaenum}

\begin{DIFnomarkup}
\hypertarget{syncerror}{}\subsection{Synchronous Errors }\label{syncerror}
\end{DIFnomarkup}

When a core runtime API is called by the user and does not execute
successfully, the core runtime returns a status that can help determine a cause of the
unsuccessful execution. Each API call discussed in this chapter defines what
constitutes a successful execution. While a few error conditions can be
generalized to a certain degree (e.g. failure in allocating system
memory) many errors can have system/implementation specific
explanations.

The HSA core runtime API defines an enumeration that captures the
result of any API function that has been executed (the only
exception to this behavior are setter/getter API that access core
runtime structures). This enumeration is of the type
\dbtt{hsa\_status\_t} and enumerates \emphld{success},
\emphld{info}, and \emphld{error}. The \emphld{info} status definition
is discussed in Section ~\ref{asyncerror}.

\emphld{Success} status is a single value,
\dbtt{HSA\_STATUS\_SUCCESS}. Description of every core runtime
API call that returns \dbtt{hsa\_status\_t} explains the
expected successful behavior for that API. The value of
\dbtt{HSA\_STATUS\_SUCCESS} is always 0.

\emphld{Error} status could be due to user input/actions that are not
allowed (e.g. negative value in a size for allocation) or systemic
errors (e.g. an asynchronous activity lead to a failure that
cascaded into a failure in this API). The constants used for error
status are restricted to the negative range of values within the
\dbtt{hsa\_status\_t} enumeration. Errors must always have a
negative value. The Name of any constant that indicates an error status is
prefixed by \dbtt{HSA\_STATUS\_ERROR}. Errors could potentially be 
implementation.

While the name of the constant in itself is informative for success,
info or error status, there may be scenarios where
\begin{inparaenum}[(i)] \item the user may request more information
about the meaning of a particular status, or, \item the return
status was implementation specific and the user needs to decode it.
\end{inparaenum} In the case of implementation specific status, the
negative number returned for error may not correspond to a
particular enumeration constant. To query additional
information on synchronous errors, the core runtime defines the
following API:

\input{APIhsa_status_query}

This API returns \dbtt{HSA\_STATUS\_SUCCESS} if one or both of the
{\itshape status\_info} and {\itshape status\_info\_string} have been 
successfully updated with information regarding the input
{\itshape input\_status}. Otherwise it returns one of the following errors:

\begin{easylist}
& \dbtt{HSA\_STATUS\_NONE} when no additional information is
available regarding the status user requested. 
& \dbtt{HSA\_STATUS\_ERROR\_INVALID\_ARGUMENT} if a NULL value is
passed for either of the arguments
\end{easylist}

\begin{DIFnomarkup}
\hypertarget{asyncerror}{}\subsection{Asynchronous Errors and
Notifications}\label{asyncerror}
\end{DIFnomarkup}

The HSA core runtime supports user-defined callbacks to handle
asynchronous errors. There are two different categories of callbacks
that can be registered by the user: \begin{inparaenum}[(i)] \item
for asynchronous information or warnings generated when the runtime
is executing, or, \item for asynchronous errors that get generated
in packet processor, or while executing a kernel \end{inparaenum}.
The core runtime supports a callback each for asynchronous errors
and notifications.
The user must use caution when using blocking functions within their
callback implementation -- a callback that does not return can
render the runtime state to be undefined. The user cannot depend on
thread local storage within the callbacks implementation and may
safely kill the thread that registers the callback. It is the user's
responsibility to ensure that the callback function is thread-safe.
The runtime does not implement any default callbacks.

\subsubsection{Asynchronous Notification of Information or
Warning}\label{asynnotif}

The information/warning status is represented by a value greater
than 0 within the \dbtt{hsa\_status\_t} enumeration. The status is
up to user interpretation and the runtime allows the user to
register a callback to take necessary action. Consider the example
where a user calls the initialize API to initialize the core runtime
and the return status is
\dbtt{HSA\_STATUS\_INFO\_ALREADY\_INITIALIZED} (to indicate that
the core runtime has already been initialized). This result may be
interpreted differently in different usage scenarios. A callback for
such notifications may be registered via \ttbf{hsa\_open} API
discussed in Section~\ref{init} or via
\ttbf{hsa\_notification\_callback\_register} API, which is defined
as follows:

\input{APIregister_notify}

The {\itshape context} parameter is used to identify a particular
runtime context that this callback is registered for. When a
callback is registered for a particular context, it will only be
invoked if the notification is for an action in that context.
Section~\ref{init} discusses the context in detail. The
\ttbf{hsa\_notification\_callback\_register} API can return one of
the following errors:
\begin{easylist}
& \dbtt{HSA\_STATUS\_ERROR\_OUT\_OF\_RESOURCES} if there is a failure
in allocation of an internal structure required by the core runtime
library in the context of registering a callback. This error may
also occur when the core runtime library needs to spawn threads or
create internal OS-specific events. 
& \dbtt{HSA\_STATUS\_ERROR\_INVALID\_ARGUMENT} 
if {\itshape info} is NULL.
\end{easylist}

One of the arguments of the notification callback is a structure
that contains notification information. The structure is defined as
follows:

\input{STRnotify_message}

\subsubsection{Asynchronous Notification of Errors}\label{asynnotif}

The HSA system can have several queues in operation and
several kernels executing from these queues asynchronously.
When any asynchronous activity generates an error, the action that
initiated the activity may have concluded. To deal with
asynchronous errors, the core runtime supports asynchronous error
callbacks. The asynchronous error callback may be registered by means of the
\ttbf{hsa\_open} API discussed in Section~\ref{init} or via
\ttbf{hsa\_error\_callback\_register} API, which is defined as
follows:

\input{APIregister_error}

Details on how association of the callback can be done with
asynchronous activities are discussed in Sections~\ref{init} and
\ref{architected_queue}. The {\itshape context} parameter is used
to identify a particular runtime context that this callback is
registered for. When a callback is registered for a particular
context, it will only be invoked if the notification is for an
action in that context. For example, if a queue was created for a
runtime context {\itshape c1} and a callback registered for a
context {\itshape c2} but not for {\itshape c1}, any error on the
queue, such as a packet processing error, will not trigger the
execution of asynchronous error callback registered for context
{\itshape c1}. This API can return one of the following errors:

\begin{easylist}
& \dbtt{HSA\_STATUS\_ERROR\_OUT\_OF\_RESOURCES} if there is a failure
in allocation of an internal structure required by the core runtime
library in the context of registering a callback. This error may
also occur when the core runtime library needs to spawn threads or
create internal OS-specific events. 
& \dbtt{HSA\_STATUS\_ERROR\_INVALID\_ARGUMENT} 
if {\itshape info} is NULL.
\end{easylist}

One of the arguments of the notification callback is a structure
that contains notification information. The structure is defined as
follows:

\input{STRerror_message}

\subsection{Asynchronous Notification Example}
This is {\color{red} work-in-progress} -- the chapter needs to be written.
